\problemname{Friends}
%\illustration{.40}{friends}{\href{}{}}
\illustration{.40}{dolores}{\href{https://www.flickr.com/photos/78865207@N05/8267108258}{CC BY-NC-SA 2.0, Dolores Umbridge by Julio Oliveiraa via Flickr}}

\noindent
High school is all about being in the coolest group of friends. Headmistress Umbridge knows this, and she also knows that knowledge is power. She has collected data on all of the $n$ students at the school, asking each of them who they are friends with. Now she has a list of responses, but she is suspicious that some of the students might not have been entirely truthful during the questioning. 

From anonymous (but highly reliable) sources, Headmistress Umbridge knows that the friendships at her school satisfy the following properties:
\begin{itemize}
  \item If $a$ is friends with $b$ then $b$ is also friends with $a$.
  \item The set of students can be partitioned into groups, such that every student participates in exactly one group, where
  \begin{itemize}
    \item each group has at least one and at most $p$ students, and
    \item for each group there are at most $q$ pairs of friends with one of them in the group, and the other one outside of it.
  \end{itemize}
\end{itemize}
\noindent
Note that two students in the same group do not have to be friends.

Umbridge has hired you to figure out whether it is possible that all students are telling the truth, or whether she can be sure that at least one student is lying, and that she therefore should put everyone in detention. Is this morally questionable? Probably. 

\section*{Input}
First a single line with three non-negative integers $n$, $p$ and $q$ as described above. Next follow $n$ lines, one for each student, starting with student $i=0$. Each such line starts with an integer $m_i$, denoting the number of friends student number $i$ claims that she has. Then follow $m_i$ distinct integers between $0$ and $n-1$, indicating who those friends are (the students are numbered from $0$ to $n-1$). 

\section*{Constraints}
We always have $1 \leq n \leq 2000$, and $p + q \leq 15$. Furthermore, it is guaranteed that $m_0 + m_1 + \ldots + m_{n-1} \leq 9000$. A student never lists herself as one of her friends. For subcases, the inputs have these further restrictions:

\begin{itemize}
    \item{\textbf{Group 1: 20 points}} $n \leq 16$
    \item{\textbf{Group 2: 37 points}} $n \leq 250$ and $q \leq 2$
    \item{\textbf{Group 3: 12 points}} $q \leq 2$
    \item{\textbf{Group 4: 31 points}} No further restrictions.
\end{itemize}

\section*{Output}
Output should contain a single word on a single line. Output ``home'' if it is possible that all students are telling the truth, or ``detention'' if you can be certain that at least one student is lying.

