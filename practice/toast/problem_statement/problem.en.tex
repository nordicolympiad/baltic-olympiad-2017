\problemname{Toast}
\illustration{.40}{toast}{\href{https://www.flickr.com/photos/purplemattfish/3081361574}{CC BY-NC-ND 2.0. ``I propose a toast to the hat,'' by purplemattfish via Flickr}}
$N$ people are sitting evenly spaced around a circular table. All of them are infinitesimally small, indeed they can be modeled as points, except that they all have pretty long arms - $D$ cm long arms, to be exact. Amer pronounces a toast, and everyone cheers! Well, everyone clinks their glass with everyone that they can reach. In other words, two people will clink glasses if their arms can reach each other across the table. In total, you hear $T$ ``{\em clink!}'' sounds as the milk glasses touch. What is the radius of the table? 

\section*{Input}
A single line containing three integers, $N$, $D$, and $T$.

\paragraph*{Constraints} We always have $2 \leq N \leq 10^4$, $100 \leq D \leq 10^9$ and $1 \leq T \leq 10^{8}$. For subcases, the inputs have these further restrictions:

\begin{description}
  \item{\textbf{29 points}} $D = 100$ and $T \leq 45$.
  \item{\textbf{71 points}} No further restrictions.
\end{description}

\section*{Output}
Output should contain two numbers $\ell$ and $h$, the lowest and highest possible radius of the table. Your answers may be off by at most $10^{-4}$. There will always be a possible radius for the table.



